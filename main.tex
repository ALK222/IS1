\documentclass{article}
\usepackage[utf8]{inputenc}
\usepackage[spanish]{babel}
\usepackage[margin=0.8in]{geometry}
\usepackage{tabularx}

\title{Tienda de ropa riconuda} 
\author{Alejandro Barrachina Argudo \\
David Cantador Piedras \\
Rodrigo Sosa Sáez }

\date{\today}

\begin{document}

\maketitle
\section*{Control de cambios} %
\noindent\begin{tabularx}{\textwidth}{ |l|l|p{5cm}|X| }
    \hline
    \textbf{Versión} & \textbf{Fecha} & \textbf{Autores} & \textbf{Descripción} \\
    \hline
                     &                &                  &                      \\
    \hline
                     &                &                  &                      \\
    \hline
                     &                &                  &                      \\
    \hline
                     &                &                  &                      \\
    \hline
                     &                &                  &                      \\
    \hline
                     &                &                  &                      \\
    \hline
                     &                &                  &                      \\
    \hline
                     &                &                  &                      \\
    \hline
                     &                &                  &                      \\
    \hline
                     &                &                  &                      \\
    \hline
\end{tabularx}


\section{introducción}
\subsection{Próposito}
Este proyecto va de una tienda de ropa, para que pueda gestionar inventario
\subsection{Alcance}
TODO
\subsection{Definiciones,acrónimos y abreviaturas}
\subsection{Referencias}
\subsection{Resumen}
\section{Descripción General }
\subsection{Perspectiva del producto}
Putos pelirrojos\\
puto alex2
\subsection{Perspectiva del producto}
\subsection{Funciones del producto}
\subsection{Características del usuario}
\subsection{Restricciones}
\subsection{Supuestos y dependencias}
\subsection{Requisitos futuros}
\section{Requisitos específicos}
\subsection{Interfaces externos}
\subsection{Requisitos funcionales}
\subsection{Requisitos de rendimiento}
\subsection{Requisitos sobre la persistencia de datos}
\subsection{Restrcicciones de diseño}
\subsection{Atributos del sistema software}

\section*{Apéndices}
\end{document}
