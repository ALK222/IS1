\section{introducción}
\subsection{Propósito}
El objetivo de este \gls{srs} es describir de manera clara y concisa las funciones y restricciones que tendrá el software.

Este documento se dirige al equipo encargado de desarrollar el software de dicho programa, al de mantenerlo y a las personas que participen en el mismo de cualquier manera. También va dirigido al cliente final para facilitar la comprensión del proyecto y su alcance, para que así puedan sugerirse cambios e ideas.
\subsection{Alcance}
El programa se denomina ``\nombredelproyecto''.

En el mundo moderno es importante tener presencia en internet para poder competir con otras plataformas, en ocasiones más asequibles. Para ello, este software pretende ofrecer a las tiendas sin soporte online una transición simple e indolora al mundo online.

``\nombredelproyecto'' será un sistema de gestión de usuarios, pedidos y empleados con una interfaz web. En este sistema tendrán que estar registrados los productos, los empleados y los usuarios para poder gestionarlas.\\
El sistema tendrá ciertas funciones específicas tales como registros de usuarios, ya sean clientes, gerentes o empleados, manejo de inventario como productos en stock, mercancía entrante y mercancía pendiente por llegar, control de precios como descuentos y promociones. El cliente por otro lado, podrá realizar pedidos, notificar desperfectos en los productos recibidos, así como realizar la devolución de los mismos y utilizar los descuentos disponibles.
Lo que no facilitará este software es el seguimiento de pedidos ni de reclamaciones. Tampoco se encargará de las trasferencias monetarias ni de compras ni de sueldos.

\glsaddall
\nocite{*}
\subsection{Definiciones, acrónimos y abreviaturas}
{\setstretch{1.2}
	\printglossary}


\subsection{Referencias}
\printbibliography[heading=none]
\subsection{Resumen}
El siguiente documento se va a estructurar en diferentes apartados, de la siguiente manera: apartado \ref{sec:desc_gen}~(\nameref{sec:desc_gen}) donde no solo se comentará la perspectiva general del producto y las funciones del mismo, sino que también se explicará sus restricciones, supuestos y dependencias y los posibles requisitos futuros así como las características del usuario; en el apartado \ref{sec:req_esp}~(\nameref{sec:req_esp}) se comentarán las interfaces externas, los requisitos funcionales y de rendimiento, así como también aparecerán los requisitos sobre la persistencia de datos, restricciones del diseño , los atributos del sistema software y un resumen de las tecnologías empleadas para su correcto funcionamiento.
