\section{introducción}
\subsection{Propósito}
El objetivo de este \gls{srs} es describir de manera clara y concisa las funciones y restricciones que tendrá el software aquí descrito.\\
Este documento se dirige al equipo encargado de desarrollar el software de dicho programa, al de mantenerlo y a las personas que participen en el mismo de cualquier manera.
\subsection{Alcance}
El programa se denomina ``\nombredelproyecto''.

En el mundo moderno, es importante tener presencia en internet para poder competir con otras plataformas, en ocasiones más asequibles. Para ello, este software pretende ofrecer a las tiendas sin soporte online una transición simple e indolora al mundo online.

``\nombredelproyecto'' será un sistema de gestión de usuarios, pedidos y empleados con una interfaz web. En este sistema tendrán que estar registrados los productos, los empleados y los usuarios para poder gestionarlas.\\
Este sistema facilitará la gestión de usuarios para la empresa, la gestión de pedidos y productos para los empleados y la gestión de los pedidos por parte de los mismos clientes.
Lo que no facilitará este software es el seguimiento de pedidos ni de reclamaciones. Tampoco se encargará de las trasferencias monetarias ni de compras ni de sueldos.

\glsaddall
\nocite{*}
{\setstretch{1.2}
    \printglossary[title=Definiciones\, acrónimos y abreviaturas, numberedsection]}



\bibliographystyle{ieeetr}
\bibliography{./references}
\subsection{Resumen}
El siguiente documento se va a estructurar en diferentes apartados, de la siguiente manera: Apartado \ref{sec:desc_gen}~(\nameref{sec:desc_gen}) donde se comentará la perspectiva general del producto y las funciones del mismo. También se explicará sus restricciones, supuestos y dependencias y los posibles requisitos futuros así como las características del usuario.
En el \ref{sec:req_esp}~(\nameref{sec:req_esp}) se comentarán las interfaces externas, los requisitos funcionales y de rendimiento. También aparecerán los requisitos sobre la persistencia de datos, restricciones del diseño , los atributos del sistema software y un resumen de las tecnologías empleadas para su correcto funcionamiento.
