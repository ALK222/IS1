\section{Mecanismos de seguimiento y control}
Para satisfacer el monitoreo del presente proyecto, se aplicarán series de inspecciones periódicas, de las cuales su tiempo de monitoreo dependerá de la evolución del software y la cantidad de fallos que este pueda producir en el tiempo. El encargado en realizar este seguimiento será un grupo especial o \gls{sqa}, el cual tendrá un jefe responsable de todos los cambios, soluciones y decisiones que se realicen en el mantenimiento, este jefe no es más que el mismo jefe del proyecto.

\subsection{Garantía de calidad y control (Plan de calidad)}
Para asegurar la calidad del producto (software), se realizarán verificaciones y revisiones técnicas formales.\\
El encargado de llevar a cabo todo esto será el jefe del grupo \gls{sqa}. Además, este grupo se encargará de ayudar a los desarrolladores para que el software alcance una calidad más alta.\\
El método que vamos a utilizar principalmente es el de realizar inspecciones. Su propósito es detectar e identificar anomalías en el producto software.\\
Esto nos servirá para:
\begin{itemize}
	\item  Verificar que el producto software satisface sus especificaciones.
	\item  Verificar que el producto software satisface los atributos de calidad especificados.
	\item Verificar que el producto software se ajuste a las regulaciones aplicables, estándares, guías, planes y procedimientos.
	\item  Identificar desviaciones con respecto a estándares y especificaciones.
	\item Recolectar datos de \gls{is}.
	\item utilizar los datos de \gls{is} recolectados para mejorar el proceso de inspección y su documentación de soporte.
\end{itemize}
Estos dos últimos puntos son opcionales, se puede llevar una eficiente gestión del proyecto con la ausencia de ellas, aun así, se recomienda utilizarlas para un control aún más óptimo del software.


Durante las inspecciones se determinarán las soluciones de las anomalías que se presenten. Esto servirá para revisar entre otras cosas: el diseño, el código fuente, y la documentación de usuarios.

Las inspecciones se llevarán a cabo de la siguiente manera:
\begin{enumerate}
	\item El responsable de llevar a cabo el producto informa del fin de un trabajo al jefe de proyecto.
	\item El jefe contacta con unos supervisores, a los cuales se les entrega el producto.
	\item Los supervisores se encargan de dar el visto bueno al proyecto durante horas.
	\item El jefe de trabajo planifica una reunión para una fecha lo más cercana posible.
	\item Uno de los supervisores actúa como testigo, este se encarga de anotar las incidencias.
	\item El creador del producto expone su producto.
	\item Los supervisores ponen pegas al producto.
	\item Cuando se descubre una incidencia el testigo las anota.
\end{enumerate}

Cuando acaba la reunión hay tres opciones: aceptar el producto sin modificaciones, rechazar el producto o aceptar el producto tras llevar a cabo unas modificaciones sin una nueva inspección.  Al finalizar la reunión, todos los participantes de la reunión firman el registro de revisión para de esta forma hacerlo oficial evitando riesgos legales o de contrato.


\subsection{Gestión y control de cambios (Plan GCS)}
Todos los miembros del proyecto podrán realizar una solicitud de cambio sobre alguno de los requerimientos del producto registrados. Para esto se deberá́ respetar el siguiente protocolo de trabajo:

\begin{itemize}
	\item Toda solicitud de cambio debe ser requerida con una justificación vía correo electrónico hacia el Jefe de Proyecto. Todo cambio realizado sin ser comunicado previamente por este medio no será aplicado a la línea base.
	\item Toda solicitud de cambio deberá ser analizada y aprobada por el Jefe de Proyecto con el soporte de la persona responsable del activo que se solicita modificar. Queda a su cargo la tarea de relevar cuál es el impacto que tendrá en el proyecto y realizar los ajustes pertinentes (reflejar en la documentación de cambios) para minimizar el mismo.
	\item Será el Jefe de Proyecto el encargado de informar al personal sobre la implementación de dicho cambio (en caso de que lo aprobara). Se utilizará la matriz que se encuentra en la sección “Plan de Gestión de la Configuración” para asignar al responsable de realizar el registro de cambio en la documentación de cambios.
	\item El jefe de proyecto evaluará las acciones correctivas y preventivas que se deberán aplicar.
	\item El líder de proyecto será responsable de verificar que se han realizado las modificaciones necesarias en los requerimientos del producto, que generaron la implementación del cambio aprobado.
	\item Luego que un cambio haya sido aprobado quedará a cargo del Líder del Proyecto la tarea de verificar que se ha registrado correctamente el cambio para mantener la integridad del proyecto.
\end{itemize}
