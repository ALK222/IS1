\section{Recursos del proyecto}

\subsection{Personal}
\begin{itemize}
	\item \textbf{\underline{Gestor superior y Gestor técnico del proyecto}}
	      \begin{itemize}
		      \item David cantador piedras
	      \end{itemize}
	\item \textbf{\underline{Profesionales}}
	      \begin{itemize}
		      \item Alejandro Barrachina Argudo
		      \item Juan Pantaleón Femenía Quevedo
		      \item David Llanes Martín
		      \item Samuel Rodríguez Moreno
		      \item Sergio Sánchez Chamizo
		      \item Rodrigo Sosa Sáez
		      \item Rodrigo Souto Santos
	      \end{itemize}
\end{itemize}

\subsection{Hardware y software}
\begin{itemize}
	\item \textbf{\underline{Software}}
	      \begin{itemize}
		      \item \href{https://discord.com}{Discord}
		      \item \href{https://www.eclipse.org/}{Eclipse IDE for Java Developers}
		      \item \href{https://github.com}{GitHub}
		      \item \href{https://drive.google.com}{Google Drive}
		      \item \href{https://meet.google.com}{Google Meet}/ \href{https://zoom.us}{Zoom}
		      \item \href{https://www.ibm.com/products/rational-software-architect-designer}{IBM Rational Software Architect}
		      \item \href{https://www.microsoft.com/es-es/microsoft-365}{Microsoft Office}
		      \item \href{https://overleaf.com}{Overleaf}
		      \item \href{https://www.microsoft.com/es-es/microsoft-365/project/project-management-software}{Microsoft Project}
		      \item \href{https://www.oracle.com/es/enterprise-manager/technologies/}{MySQL/Oracle DBMS}
		      \item \href{https://www.microsoft.com/es-es/windows-server}{Windows Server}
	      \end{itemize}
	\item \textbf{\underline{Hardware}}
	      \begin{itemize}
		      \item Servidores rack
		      \item Puestos de trabajo
		      \item Impresoras
		      \item Fax y scanner
	      \end{itemize}
\end{itemize}

\subsection{Lista de recursos}

\begin{itemize}
	\item \textbf{\underline{Comunicación}}
	      \begin{itemize}
		      \item \textbf{Google meet/Zoom:} medio empleado para la comunicación con el stakeholder.
		      \item\textbf{Discord:} Medio empleado para la comunicación entre el personal del proyecto.
	      \end{itemize}
	\item \textbf{\underline{Entornos de almacenamiento y repositorio}}
	      \begin{itemize}
		      \item\textbf{GitHub:} medio empleado como repositorio, control de versiones y gestor de la configuración del proyecto.
		      \item\textbf{Google Drive:} medio empleado para el almacenamiento online de documentos relacionados con el proyecto.
	      \end{itemize}
	\item \textbf{\underline{Edición de documentos:}}
	      \begin{itemize}
		      \item\textbf{Microsoft Office:} paquete de programas empleado para la creación y edición de documentos  (uso mayoritario de Word)
		      \item\textbf{Overleaf:} editor online para la maquetación y edición de documentos en \LaTeX
	      \end{itemize}
	\item \textbf{\underline{Entornos de desarrollo:}}
	      \begin{itemize}
		      \item\textbf{MySQL/Oracle DBMS:} Entorno de desarrollo y gestión de la \gls{bd} asociada al producto.
		      \item\textbf{Eclipse IDE for Java Developers:} entorno de desarrollo principal para la creación del producto en lenguaje java.
		      \item\textbf{Windows Server:} entorno de gestión y control de servidores.
	      \end{itemize}
	\item \textbf{\underline{Entorno de planificación de proyectos}}
	      \begin{itemize}
		      \item\textbf{Microsoft Project:} entorno empleado para establecer la gestión del proyecto.
	      \end{itemize}
	\item \textbf{\underline{Entorno de diseño}}
	      \begin{itemize}
		      \item\textbf{IBM Rational Software Architect:} entorno empleado para diseñar la interfaz del producto.
	      \end{itemize}
	\item \textbf{\underline{Hardware}}
	      \begin{itemize}
		      \item\textbf{Servidores Rack:} Permiten que los clientes puedan acceder al servidor en cualquier momento, almacenar la \gls{bd} y hacer copias de seguridad.
	      \end{itemize}
\end{itemize}
