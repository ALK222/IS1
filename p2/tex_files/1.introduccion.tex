\section{Introducción}
\subsection{Propósito del plan}
\begin{itemize}
	\item Tener un texto formal con las planificaciones, estrategias y acciones estimadas para realizar el proyecto de forma ordenada y completando plazos.
	\item Tener en cuenta plazos, riesgos, actividades, equipo, costes, cambios, etc. Todo de forma ordenada para que el trabajo que se ejecute siga la dirección que necesitamos.
\end{itemize}

\subsection{Ámbito del proyecto y objetivos}
\subsubsection{Declaración del proyecto}
Esta aplicación está dirigida a bancos que quieran dar soporte online a sus clientes en sus operaciones del día a día como gestionar sus ingresos, recibir su nómina o comprar y vender acciones.
\subsubsection{Funciones principales}
El software permitirá la gestión de tareas básicas para el usuario y para el propio banco. Estas acciones incluyen la gestión de cuentas, tarjetas bancarias, prestamos, acciones y la actividad general de la cuenta.
\subsubsection{Aspectos de rendimiento}
Todas las funciones a excepción del renderizado de la interfaz están gestionadas por el servidor del propio banco, por lo tanto estará limitado por la potencia del servidor, las conexiones concurrentes al mismo y la potencia del ordenador del usuario y de su velocidad de red.
\subsubsection{Restricciones y técnicas de gestión}
En las funciones principales de gestión de usuarios se pedirá el DNI y la contraseña del usuario.

\noindent La baja de dicho usuario se ejecutará si y solo si el usuario ya está registrado en la \gls{bd}.

Las funciones más importantes de un usuario requerirán de su firma electrónica, para así verificar con más seguridad que no se trata de una suplantación y evitar problemas tanto al cliente como al banco.

Las funciones principales de gestión del gestor, el alta de un gestor y para la baja de un gestor al igual que las anteriores se necesita que el gestor esté registrado en la \gls{bd}.

\subsection{Modelo de proceso}
Se va a usar el modelo de proceso \gls{rup}, basado en componentes conectados a través de interfaces y dirigido por casos de uso. En espiral, de forma iterativa se siguen las siguientes fases de desarrollo:
\begin{itemize}
	\item\textbf{Requisitos (Comunicación):} mediante la comunicación con el cliente se realiza una especificación de los requisitos que deberá cumplir la aplicación.
	\item\textbf{Análisis (Planteamiento):} se crea un plan de proyecto evaluando los riesgos, se definen las características y funciones mediante casos de uso preliminares. Se crea una arquitectura basada en subsistemas o componentes muy genérica, que deberá ser desarrollada durante la fase de diseño mediante modelado.
	\item\textbf{Diseño (Modelado):} Se mejoran y amplían los casos de uso de la fase de análisis. Se modelan los subsistemas y se realiza la línea de base de la arquitectura del sistema.
	\item\textbf{Implementación (Construcción): }Se desarrollan o adquieren los componentes del software necesario para completar los casos de uso necesarios. Se completan los modelos (requisitos, casos de uso, etc.) para poder implementarlos en código fuente y se efectúan pruebas unitarias de cada uno de los componentes. También se deben realizar pruebas de integración para comprobar que la aplicación funciona.
	\item\textbf{Prueba (Despliegue):} los usuarios finales o una muestra de estos realizan pruebas Beta\footnote{Beta: fase de desarrollo de un software en la que se realiza una eliminación de errores de forma activa}, que reportan los defectos y carencias del software. En esta fase se generan también los manuales de usuario, paquetes de instalación. plataformas de ayuda necesarias para el lanzamiento.
\end{itemize}

Al finalizar la fase de despliegue tenemos una aplicación funcional que puede ser usada. Se incrementa la versión del software y se realiza la siguiente iteración del proceso. Nosotros nos encontramos en esta entrega en la fase de elaboración (que incluye parte de los requisitos, pero sobre todo análisis) de la primera iteración.
