\section{Estrategia de gestión de riesgo}
En este apartado vamos a tratar la gestión de riesgo de un banco online, la cual consiste en tratar todos los posibles riesgos que pueden darse y cómo tratarlos de la mejor manera posible incluso antes de que sucedan,  se va a utilizar una estrategia de gestión proactiva es decir, anticiparse a los problemas o crisis que pueden suceder para minimizar los problemas.\\
Para analizar los riesgos vamos a tratar todos los casos de uso que nos han tocado.

\subsection{Introducción: Estudio de los riesgos}
A continuación vamos a indicar todos los posibles riesgos que pueden ocurrir y por cada uno de ellos, indicaremos la frecuencia con la que pueden darse, la descripción de dicho riesgo, la severidad o cuánto de grave es ese problema para el banco y sus \textbf{Consecuencia}s.
Para ello vamos a utilizar una tabla SQAS-SEI.
\begin{table}[H]
	\centering
	\small
	\begin{tabularx}{\textwidth}{|>{\columncolor[gray]{0.8}}p{3cm}|p{1.9cm}|p{3cm}|p{2.1cm}|X|}
		\hline
		\rowcolor{gray}
		\textbf{Riesgo}                         & \textbf{Frecuencia} & \textbf{Descripción}                                                            & \textbf{Severidad} & \textbf{\textbf{Consecuencia}s}                                                                                                                                                                                                                   \\
		\hline
		Error al iniciar sesión                 & Frecuente           & El usuario no puede iniciar sesión                                              & Serio              & Al no poder iniciar sesión, no puede realizar ningún tipo de acción dentro del banco..                                                                                                                                                            \\
		\hline
		Error al cerrar sesión                  & Frecuente           & El usuario no puede cerrar sesión                                               & Despreciable       & No es un problema grave ya que aunque no puede cerrar sesión, no afecta a nada ya que dicho cliente puede realizar todo tipo de acción dentro del banco.                                                                                          \\
		\hline
		Error al crear una cuenta al cliente    & Probable            & El gestor no puede crear una cuenta al cliente                                  & Menor              & Es similar al no iniciar sesión ya que no puede realizar ningún tipo de acción dentro del banco..                                                                                                                                                 \\
		\hline
		Error al eliminar una cuenta al cliente & Ocasional           & El gestor no puede eliminar la cuenta de un cliente                             & Menor              & Es algo más importante que no poder cerrar sesión ya que un banco se lleva un porcentaje mientras seas de ese banco, con lo que no poder eliminar tu cuenta afecta directamente al saldo del cliente el cual quiere darse de baja de dicho banco. \\
		\hline
		Error al seleccionar un cliente         & Probable            & El gestor no puede acceder a un cliente para modificar sus datos en la \gls{bd} & Serio              & Imposibilita al gestor modificar tus datos o actualizarlos con lo cual es algo serio ya que el cliente siempre quiere tener sus datos actualizados cuanto antes y no tener errores en sus cuentas.                                                \\
		\hline
	\end{tabularx}
	\caption{Tabla SQA-SEI 1}
\end{table}


\begin{table}[H]
	\centering
	\small
	\begin{tabularx}{\textwidth}{|>{\columncolor[gray]{0.8}}p{3cm}|p{1.9cm}|p{3cm}|p{2.1cm}|X|}
		\hline
		Error al cargar tarjeta del cliente         & Remoto     & El cliente no puede cargar su tarjeta de débito con el dinero que posee en su cuenta bancaria, lo cual suponemos que ha sido por un error del banco. & Crítico & Es un problema crítico casi catastrófico, estamos hablando de un dinero que le pertenece con lo que no poder acceder a su dinero afectaría muy negativamente a la confianza de la persona con el banco.                                                                                             \\
		\hline
		Error al descargar la tarjeta del cliente   & Remoto     & El cliente no puede descargar su tarjeta débito con el dinero que posee en su tarjeta                                                                & Menor   & No es algo muy grave, ya que puedes seguir utilizando el dinero que tienes en la tarjeta y no habría problemas mayores.                                                                                                                                                                             \\
		\hline
		Error al contactar con un gestor            & Remoto     & El cliente no es capaz de contactar con el gestor a través de la página de contacto diseñada para ello                                               & Menor   & Las \textbf{Consecuencia}s no son demasiado relevantes, sin embargo, puede tener un problema importante y si no contacta con un gestor puede agravarse aún más, aparte de que el cliente perderá su confianza con el banco.                                                                         \\
		\hline
		Error al cambiar el pin de la tarjeta       & Improbable & El gestor no puede cambiar el pin de la tarjeta de un cliente                                                                                        & Menor   & No supone un problema tan grave ya que simplemente puede seguir usando la tarjeta de forma normal sin cambiar de pin ya que suponemos que conoce su pin anterior, en el caso de que lo quiera cambiar porque se haya olvidado puede preguntar al banco cuál era su pin, sin necesidad de cambiarlo. \\
		\hline
		Error al bloquear la tarjeta del cliente    & Ocasional  & El cliente no puede bloquear su tarjeta                                                                                                              & Crítico & El cliente no puede bloquear su tarjeta, y si suponemos que se la han robado es aún peor ya que este cliente no estará contento debido a que su tarjeta no pudo bloquearla y generará una gran desconfianza y una mala reputación hacia el banco.                                                   \\
		\hline
		Error al desbloquear la tarjeta del cliente & Remoto     & El cliente no puede desbloquear su tarjeta                                                                                                           & Menor   & La única \textbf{Consecuencia} relevante es que dicho cliente no podrá usar su tarjeta, lo cual no es nada positivo y es similar al caso de no poder sacar dinero.                                                                                                                                  \\
		\hline
	\end{tabularx}
	\caption{Tabla SQA-SEI 2}
\end{table}

\begin{table}[H]
	\centering
	\small
	\begin{tabularx}{\textwidth}{|>{\columncolor[gray]{0.8}}p{3cm}|p{1.9cm}|p{3cm}|p{2.1cm}|X|}
		\hline
		Error al ingresar nómina en cuenta                                                                                                                                                   & Probable  & El cliente no puede ingresar su nómina en su cuenta personal                                                                                                                                                                                     & Crítico      & Un cliente siempre quiere tener fiabilidad con su banco, y que haya errores a la hora de ingresar la nómina es algo bastante crítico para la confianza de las personas con el banco.                                               \\
		\hline
		Error al cambiar el titular de la tarjeta                                                                                                                                            & Ocasional & El gestor no puede cambiar el titular de una cuenta                                                                                                                                                                                              & Despreciable & No afecta en nada al transcurso del dinero en sí con lo cual no es un problema grave ni prioritario.                                                                                                                               \\
		\hline
		Error al realizar una transferencia                                                                                                                                                  & Remoto    & El cliente tiene errores al intentar realizar una transferencia nacional/ internacional                                                                                                                                                          & Catastrófico & Es un problema súper grave ya que puede que haya pérdida dinero si hay fallos en las transferencias, y muchas veces las transferencias implican gran cantidad de dinero con lo que afecta a otro negocios y dificulta la economía. \\
		\hline
		Error al cambiar tu clave de seguridad                                                                                                                                               & Ocasional & El gestor no puede cambiar la clave de seguridad de la tarjeta del cliente                                                                                                                                                                       & Menor        & Al no cambiar la clave de seguridad suponemos que sí puede bloquearla con lo que, por ejemplo si se la han robado bloquea la tarjeta no podrán usarla sencillamente, por lo que no es un problema crítico                          \\
		\hline
		Error al solicitar un préstamo\newline
		---------\newline
		Error al aprobar un préstamo\newline
		---------\newline
		Error al denegar un préstamo                                                                                                                                                         & Probable  & El cliente no puede solicitar un préstamo,
		estamos suponiendo que no es problema del cliente sino fallo en el sistema de banco al no poder aprobar un préstamo, con lo cual estos tres casos de uso son prácticamente idénticos & Serio     & Muchos clientes solicitan préstamos y es algo bastante común en los bancos, que cuya principal fuente de ingresos de estos es el interés que reciben de los préstamos con lo que que haya fallos en un proceso tan importante es bastante serio.                                                                                                                                                                                                                                                     \\
		\hline
	\end{tabularx}
	\caption{Tabla SQA-SEI 3}
\end{table}

\begin{table}[H]
	\centering
	\small
	\begin{tabularx}{\textwidth}{|>{\columncolor[gray]{0.8}}p{3cm}|p{1.9cm}|p{3cm}|p{2.1cm}|X|}
		\hline
		Error al comprar acciones por parte del cliente & Improbable & El cliente no puede comprar acciones debido a un fallo en el sistema del banco & Crítico      & Comprar y vender acciones es algo súper común que mueve muchísimo dinero y prácticamente con lo que se mueven muchísimas empresas grandes, no es algo catastrófico ya que al poder comprar acciones el cliente no pierde dinero, pero tampoco lo gana, con lo que perderá mucha credibilidad en el banco en estos temas si fallan. \\
		\hline
		Error al vender acciones por parte del cliente  & Probable   & El cliente no puede vender acciones debido a un fallo en el sistema del banco  & Catastrófico & No poder vender acciones es un problema muy gordo, como he dicho antes, se mueve mucho dinero en la bolsa y al no poder vender acciones, el cliente puede perder mucho dinero por un fallo del banco, esto generaría una desconfianza brutal en todos los aspectos del banco.                                                      \\
		\hline
	\end{tabularx}
	\caption{Tabla SQA-SEI 4}
\end{table}
\newpage
\subsection{Priorización de riesgos del proyecto}
Ahora vamos a utilizar una tabla para realizar la priorización de riesgos del proyecto, la cual la rellenamos con todos los riesgos posibles según su probabilidad y su severidad.\\
Los niveles de riesgo son:
\begin{itemize}
	\item \textbf{T: Tolerable.} Si sucede, no importa.
	\item\textbf{L: Bajo.} Si sucede, los efectos son asumibles.
	\item\textbf{M: Medio.} Si sucede, afecta a los objetivos, costes o planificación. Debería controlarse.
	\item\textbf{H: Alto.} Si sucede tiene una grave trascendencia. Debería controlarse, supervisarse y tener planes de contingencia.
	\item\textbf{IN: Intolerable.} No puede obviarse su gestión bajo ningún concepto.

\end{itemize}
\begin{adjustwidth}{0cm}{-8cm}
	\begin{multicols}{2}
		\begin{table}[H]
			\fontsize{8}{9}\selectfont
			\centering
			\begin{tabularx}{0.7\textwidth}{|>{\centering}X|>{\centering}X|>{\centering}X|>{\centering}X|>{\centering}X|X|}
				\hline
				\textbf{Relación de \textbf{Probabilidad} y Severidad} & Frecuente                                         & Probable                                                                          & Ocasional                                             & Remoto                                                                       & Improbable                                \\
				\hline
				Catastrófico                                           & \cellcolor{riskred}                               & \cellcolor{riskred}   E. bloquear tarjeta del cliente \newline E. vender acciones & \cellcolor{riskred}                                   & \cellcolor{riskyellow} E. realizar una transferencia                         & \cellcolor{riskgreen} E. vender acciones  \\
				\hline
				Crítico                                                & \cellcolor{riskred}                               & \cellcolor{riskred}                                                               & \cellcolor{riskyellow}                                & \cellcolor{riskgreen} E. cargar tarjeta \newline	E. ingresar nómina en cuenta & \cellcolor{riskpurple}E. comprar acciones \\
				\hline
				Serio                                                  & \cellcolor{riskyellow}                            & \cellcolor{riskyellow} E. seleccionar cliente\newline
				E. solicitar un préstamo\newline
				E. aprobar un préstamo\newline
				E. denegar un préstamo\newline                         & \cellcolor{riskgreen} E. eliminar cuenta          & \cellcolor{riskpurple}                                                            & \cellcolor{riskblue}                                                                                                                                                             \\
				\hline
				Menor                                                  & \cellcolor{riskgreen} E. iniciar sesión           & \cellcolor{riskgreen}E. crear cuenta                                              & \cellcolor{riskpurple}E. cambiar clave de seguridad   & \cellcolor{riskblue} E. descargar tarjeta\newline
				E. contactar con gestor\newline
				E. desbloquear tarjeta del cliente                     & \cellcolor{riskblue} E. cambiar pin de la tarjeta                                                                                                                                                                                                                                                                        \\
				\hline

				Despreciable                                           & \cellcolor{riskgreen} E. cerrar sesión            & \cellcolor{riskpurple} \cellcolor{riskblue}                                       & \cellcolor{riskblue} E. cambiar titular de la tarjeta & \cellcolor{riskblue}                                                         & \cellcolor{riskblue}                      \\
				\hline
			\end{tabularx}
			\caption{Tabla de Riesgos}
			\label{tab:risk}
		\end{table}
		\vfill
		\null
		\vfill

		\fbox{\begin{tabular}{ll}
				\textcolor{riskblue}{$\blacksquare$}   & Tolerable   \\
				\textcolor{riskpurple}{$\blacksquare$} & Bajo        \\
				\textcolor{riskgreen}{$\blacksquare$}  & Medio       \\
				\textcolor{riskyellow}{$\blacksquare$} & Alto        \\
				\textcolor{riskred}{$\blacksquare$}    & Intolerable \\
			\end{tabular}}
	\end{multicols}
\end{adjustwidth}

\newpage
\subsubsection{Exposición al riesgo}

Ahora ordenaremos los riesgos de mayor a menor prioridad según la exposición al riesgo

\begin{table}[H]
	\centering
	\begin{tabularx}{0.6\textwidth}{|X|X|}
		\hline
		\multicolumn{2}{|>{\hsize=\dimexpr2\hsize+2\tabcolsep+\arrayrulewidth\relax}X|}{\textbf{Error al seleccionar cliente \newline
			Error al vender acciones\newline
			Error al bloquear la tarjeta del cliente}}
		\\
		\hline
		\textbf{Probabilidad}    & Probable  \\
		\hline
		\textbf{Consecuencia}    & Serio     \\
		\hline
		\textbf{Nivel de riesgo} & (5-0.5)\% \\
		\hline
	\end{tabularx}
	\caption{Tabla de nivel de riesgo 1}
\end{table}

\begin{table}[H]
	\centering
	\begin{tabularx}{0.6\textwidth}{|X|X|}
		\hline
		\multicolumn{2}{|>{\hsize=\dimexpr2\hsize+2\tabcolsep+\arrayrulewidth\relax}X|}{\textbf{Error al seleccionar cliente \newline
			Error al solicitar un préstamo \newline
			Error al aprobar un préstamo \newline
			Error al denegar un préstamo}}
		\\
		\hline
		\textbf{Probabilidad}    & Probable  \\
		\hline
		\textbf{Consecuencia}    & Serio     \\
		\hline
		\textbf{Nivel de riesgo} & (3-0.3)\% \\
		\hline
	\end{tabularx}
	\caption{Tabla de nivel de riesgo 2}
\end{table}

\begin{table}[H]
	\centering
	\begin{tabularx}{0.6\textwidth}{|X|X|}
		\hline
		\multicolumn{2}{|>{\hsize=\dimexpr2\hsize+2\tabcolsep+\arrayrulewidth\relax}X|}{\textbf{Error al iniciar sesión}}
		\\
		\hline
		\textbf{Probabilidad}    & Frecuente \\
		\hline
		\textbf{Consecuencia}    & Menor     \\
		\hline
		\textbf{Nivel de riesgo} & $>$2\%    \\
		\hline
	\end{tabularx}
	\caption{Tabla de nivel de riesgo 3}
\end{table}

\begin{table}[H]
	\centering
	\begin{tabularx}{0.6\textwidth}{|X|X|}
		\hline
		\multicolumn{2}{|>{\hsize=\dimexpr2\hsize+2\tabcolsep+\arrayrulewidth\relax}X|}{\textbf{Error al crear cuenta}}
		\\
		\hline
		\textbf{Probabilidad}    & Probable  \\
		\hline
		\textbf{Consecuencia}    & Menor     \\
		\hline
		\textbf{Nivel de riesgo} & (2-0.2)\% \\
		\hline
	\end{tabularx}
	\caption{Tabla de nivel de riesgo 4}
\end{table}

\begin{table}[H]
	\centering
	\begin{tabularx}{0.6\textwidth}{|X|X|}
		\hline
		\multicolumn{2}{|>{\hsize=\dimexpr2\hsize+2\tabcolsep+\arrayrulewidth\relax}X|}{\textbf{Error al cerrar sesión}}
		\\
		\hline
		\textbf{Probabilidad}    & Frecuente    \\
		\hline
		\textbf{Consecuencia}    & Despreciable \\
		\hline
		\textbf{Nivel de riesgo} & $>$1\%       \\
		\hline
	\end{tabularx}
	\caption{Tabla de nivel de riesgo 5}
\end{table}


\begin{table}[H]
	\centering
	\begin{tabularx}{0.6\textwidth}{|X|X|}
		\hline
		\multicolumn{2}{|>{\hsize=\dimexpr2\hsize+2\tabcolsep+\arrayrulewidth\relax}X|}{\textbf{Error al eliminar cuenta}}
		\\
		\hline
		\textbf{Probabilidad}    & Ocasional    \\
		\hline
		\textbf{Consecuencia}    & Serio        \\
		\hline
		\textbf{Nivel de riesgo} & (0.3-0.03)\% \\
		\hline
	\end{tabularx}
	\caption{Tabla de nivel de riesgo 6}
\end{table}

\begin{table}[H]
	\centering
	\begin{tabularx}{0.6\textwidth}{|X|X|}
		\hline
		\multicolumn{2}{|>{\hsize=\dimexpr2\hsize+2\tabcolsep+\arrayrulewidth\relax}X|}{\textbf{Error al cambiar clave de seguridad}}
		\\
		\hline
		\textbf{Probabilidad}    & Ocasional    \\
		\hline
		\textbf{Consecuencia}    & Menor        \\
		\hline
		\textbf{Nivel de riesgo} & (0.2-0.02)\% \\
		\hline
	\end{tabularx}
	\caption{Tabla de nivel de riesgo 7}
\end{table}

\begin{table}[H]
	\centering
	\begin{tabularx}{0.6\textwidth}{|X|X|}
		\hline
		\multicolumn{2}{|>{\hsize=\dimexpr2\hsize+2\tabcolsep+\arrayrulewidth\relax}X|}{\textbf{Error al cambiar titular de la tarjeta}}
		\\
		\hline
		\textbf{Probabilidad}    & Ocasional    \\
		\hline
		\textbf{Consecuencia}    & Despreciable \\
		\hline
		\textbf{Nivel de riesgo} & (0.1-0.01)\% \\
		\hline
	\end{tabularx}
	\caption{Tabla de nivel de riesgo 8}
\end{table}

\begin{table}[H]
	\centering
	\begin{tabularx}{0.6\textwidth}{|X|X|}
		\hline
		\multicolumn{2}{|>{\hsize=\dimexpr2\hsize+2\tabcolsep+\arrayrulewidth\relax}X|}{\textbf{Error al realizar una transferencia}}
		\\
		\hline
		\textbf{Probabilidad}    & Remoto          \\
		\hline
		\textbf{Consecuencia}    & Serio           \\
		\hline
		\textbf{Nivel de riesgo} & (0.03-0.0003)\% \\
		\hline
	\end{tabularx}
	\caption{Tabla de nivel de riesgo 9}
\end{table}

\begin{table}[H]
	\centering
	\begin{tabularx}{0.6\textwidth}{|X|X|}
		\hline
		\multicolumn{2}{|>{\hsize=\dimexpr2\hsize+2\tabcolsep+\arrayrulewidth\relax}X|}{\textbf{Error al descargar tarjeta \newline
			Error al contactar con gestor \newline
			Error al desbloquear tarjeta del cliente}}
		\\
		\hline
		\textbf{Probabilidad}    & Remoto          \\
		\hline
		\textbf{Consecuencia}    & Menor           \\
		\hline
		\textbf{Nivel de riesgo} & (0.02-0.0002)\% \\
		\hline
	\end{tabularx}
	\caption{Tabla de nivel de riesgo 10}
\end{table}

\begin{table}[H]
	\centering
	\begin{tabularx}{0.6\textwidth}{|X|X|}
		\hline
		\multicolumn{2}{|>{\hsize=\dimexpr2\hsize+2\tabcolsep+\arrayrulewidth\relax}X|}{\textbf{Error al cargar tarjeta \newline
			Error al ingresar nómina en cuenta}}
		\\
		\hline
		\textbf{Probabilidad}    & Remoto          \\
		\hline
		\textbf{Consecuencia}    & Crítico         \\
		\hline
		\textbf{Nivel de riesgo} & (0.04-0.0004)\% \\
		\hline
	\end{tabularx}
	\caption{Tabla de nivel de riesgo 11}
\end{table}

\begin{table}[H]
	\centering
	\begin{tabularx}{0.6\textwidth}{|X|X|}
		\hline
		\multicolumn{2}{|>{\hsize=\dimexpr2\hsize+2\tabcolsep+\arrayrulewidth\relax}X|}{\textbf{Error al cambiar pin de la tarjeta}}
		\\
		\hline
		\textbf{Probabilidad}    & Improbable          \\
		\hline
		\textbf{Consecuencia}    & Menor               \\
		\hline
		\textbf{Nivel de riesgo} & (0.0002-0.000002)\% \\
		\hline
	\end{tabularx}
	\caption{Tabla de nivel de riesgo 12}
\end{table}

\begin{table}[H]
	\centering
	\begin{tabularx}{0.6\textwidth}{|X|X|}
		\hline
		\multicolumn{2}{|>{\hsize=\dimexpr2\hsize+2\tabcolsep+\arrayrulewidth\relax}X|}{\textbf{Error al comprar acciones}}
		\\
		\hline
		\textbf{Probabilidad}    & Improbable         \\
		\hline
		\textbf{Consecuencia}    & Crítico            \\
		\hline
		\textbf{Nivel de riesgo} & (0.0004-0.00004)\% \\
		\hline
	\end{tabularx}
	\caption{Tabla de nivel de riesgo 13}
\end{table}


A partir del listado anterior de los riesgos se decide qué riesgos se van a tratar, que serán
aquellos que tengan una mayor exposición. El resto de riesgos son asumidos dado la baja
probabilidad de que ocurran.

\subsection{Plan de gestión de riesgo: Reducción, supervisión y gestión de riesgo}
A continuación vamos a realizar el plan de gestión de los 2 casos de más exposición del anterior apartado (el resto de los casos son similares en muchos de los puntos), que consiste en tratar 3 puntos para cada uno, la reducción del riesgo en para que no suceda, la supervisión del mismo para comprobar si ha sucedido o no y su plan de contingencia en caso de que dicho riesgo se haya producido.\\
Como he dicho antes todos los casos se pueden resolver de una manera similar con lo que vamos a exponer los dos ejemplos de mayor exposición al riesgo.
\subsubsection{Riesgo 1: error al vender acciones}
\textbf{Reducción\_1:} La manera de reducir los riesgos está directamente implicada con el sistema informático ya que estamos tratando con un banco online, con lo cual toda la posible reducción de riesgos solo es posible en el terreno de la programación, esto se podría aplicar al resto de los casos también, se podría resolver mejorando todo el ámbito informático, contratando a muy buenos programadores y buenos equipos de trabajo.\\
También con el objetivo de que el problema no sea agrave mucho, solucionar los pequeños errores y darles importancia para que no se produzca un caos total, en este caso si vemos que sólo una persona de cien ha tenido problemas al vender acciones, aunque sea una persona, es probable que más personas se pueden ver afectadas más tarde, con lo que lo lógico sería solucionar dicho problema por pequeño que sea cuanto antes.

\textbf{Supervisión\_1:} Para saber si este riesgo ha sucedido podríamos saberlo de varias formas, que el ordenador detecte que hay ciertos errores al intentar vender acciones o que sea la propia persona la que pueda hacer un reporte del error, en definitiva se debería mejorar la infraestructura de la empresa para que haya métodos más efectivos a la hora de detectar errores y poder resolverlos con mucha más eficacia, y aunque no haya habido errores revisar todo el código para poder prever errores futuros.

\textbf{Plan de contingencia\_1:} Una vez ha tenido lugar este error, una de las cosas más efectivas que podríamos hacer sería tener una copia de seguridad de todo, pongámonos en el caso de que al vender acciones no recibe el dinero de dichas acciones debido a un error, con lo que al cliente le saldría que ha vendido esas acciones pero no ha recibido nada a cambio, se puede solucionar mediante una copia de seguridad en todo momento, ya que el objetivo principal es que el cliente no se vea afectado y en segundo lugar se hace una copia para volver al estado anterior del cliente para que nadie más de la empresa se vea afectado tampoco.


\subsubsection{Riesgo 2: error al bloquear la tarjeta del cliente}
\textbf{Reducción\_2:} Al igual que en el caso anterior, la gestión de la reducción de un riesgo se podría subsanar mediante una buena infraestructura informática y de esta manera evitar problemas de programación lo cual se puede resolver empleando gran cantidad de dinero a su soporte, tanto un buen mantenimiento del sistema y por supuesto un gran trato a los trabajadores (lo cual es imprescindible en una empresa) para que se sientan cómodos y puedan realizar su trabajo con mayor comodidad y eficacia.

\textbf{Supervisión\_2:} A la hora de supervisar el correcto funcionamiento de esta función, se deberá tener un buen plan de supervisión, como por ejemplo, una rápida respuesta del servidor y un buen sistema de detección de fallos con lo que es fácil saber si algo falla rápidamente ya que el ordenador nos indica si ha habido un problema que no se ha podido resolver o si hay errores e irregularidades en las cuentas de los clientes(en este caso, por ejemplo, mucho gasto de dinero porque les han robado la tarjeta y se están gastando el dinero de nuestro cliente) y también serán ellos mismos los que nos informen de ciertos problemas ya que les afecta directamente.\\
Lógicamente cada función del banco deberá ser supervisada de manera frecuente por los programadores aunque no hayan dado fallos para ver si es posible arreglar un problema que aún no ha ocurrido.

\textbf{Plan de contingencia\_2:} En el caso de que este error de a lugar, lo primero que se deberá conseguir es resolver el problema cuanto antes y de una manera rápida, mientras tanto, pongámonos en el caso de que a este cliente le han robado la tarjeta y se están gastando su dinero, pues mientras se intenta solucionar el problema, hay métodos en los bancos en los cuales no puedes sacar una gran cantidad de dinero de golpe y si ése es el caso, se consulta con el dueño de la tarjeta para ver si está de acuerdo, con lo que aunque le roben,  el gasto al menos será mínimo, y por supuesto todo gasto perdido aunque sea pequeño, lo subsanará el banco ya que el error es del mismo y a un cliente hay que garantizarle confianza y estabilidad.
Con estos pasos, se reducirá al mínimo el problema, y una vez esté solucionado mejorar dicha función y supervisarla para que no vuelva a ocurrir.
