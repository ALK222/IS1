\section{introducción}
\subsection{Propósito}
Es un proyecto dirigido a la administración de forma online de una tienda de ropa. Se trata de un programa que permitirá la interacción online entre la tienda y los clientes, proporcionandoles la posibilidad de realizar la compra o reserva de productos entre otras cosas, así como la disponibilidad de diversos descuentos y promociones.
Además permitirá a la tienda de ropa gestionar sus productos. 
\subsection{Alcance}
El producto abarca todas las necesidades que puede llegar a tener una tienda de ropa. La aplicación almacenará en la Base de Datos \gls{(bd)} los diferentes usuarios que se registren (tanto clientes, administradores como empleados), con sus datos correspondientes. 
Permitirá a los administradores gestionar los productos en stock asi como los descuentos y promociones. El cliente por otro lado, podrá realizar pedidos, notificar desperfectos en los productos recibidos así como realizar la devolución de los mismos y utilizar los descuentos disponibles. 

\glsaddall
\nocite{*}
{\setstretch{1.2}
    \printglossary[title=Definiciones\, acrónimos y abreviaturas, numberedsection]}

\subsection{Referencias}
\subsection{Resumen}
El siguiente documento se va a estructurar en diferentes apartados, de la siguiente manera: Apartado 2 (Descripción General) donde se comentará la perspectiva general del producto y las funciones del mismo. También se explicará sus restricciones, supuestos y dependencias y los posibles requisitos futuros asi como las características del usuario. 
En el Apartado 3 (Requisitos específicos) se comentarán las interfaces externas, los requisitos funcionales y de rendimiento. También aparecerán los requisitos sobre la persistencia de datos, restricciones del diseño , los atributos del sistema software y un resumen de las tecnologías empleadas para su correcto funcionamiento.