\section{Descripción General }
\label{sec:desc_gen}
\subsection{Perspectiva del producto}
El proyecto software está ideado para integrarse con todos los productos de la tienda, con los empleados, los clientes y el sistema.
En un principio el producto estará adaptado para gestionar los trámites de compra entre los clientes y la propia tienda de ropa. También se adaptará para un acceso multiplataforma.
Desde un punto de vista de la adaptación dentro del propio software que usa la tienda de ropa, la aplicación será compatible con cualquier sistema operativo.
Respecto a los backups, se hará los que nos proporcione el propio sistema de la \gls{bd} ya que será un sistema centralizado para cada tienda de ropa.
\subsection{Funciones del producto}
Desde un punto de vista más general, la solución permitirá a la tienda de ropa unificar todo el portal
informativo facilitando el acceso de la información pertinente a los usuarios.\\
Los subsistemas serían:

\begin{itemize}
    \item\textbf{Cliente: }Permitirá la creación, eliminación de un cliente, cambiar contraseña, cambiar nombre, Login y Logout, buscar productos y añadirlos al carrito, aplicar descuentos, informar de desperfectos, comprobar el estado del pedido, comprar productos, reservarlos y devolverlos. Puede ser accesible por el propio usuario que ejerce de cliente.
    \item\textbf{Administrador: }Permitirá tanto la creación cómo la eliminación de un administrador, sancionar a un cliente, cambiar contraseña, cambiar nombre, Login y Logout, gestión de inventario cómo actualizar Stock, quitar productos, añadir promociones y quitarlas. Sólo es accesible por el propio usuario que ejerce de administrador, es decir, un administrador sólo puede realizar operaciones sobre el propio perfil de administrador.
    \item\textbf{Empleado: }Permitirá la creación, eliminación de un empleado, cambiar contraseña, cambiar nombre, Login y Logout. Puede ser accesible por el propio usuario que ejerce de empleado.
\end{itemize}

\subsection{Características del usuario}
Este software tiene como objetivo todas las tiendas de ropa existentes, aun así, pensamos
que aquellas tiendas de tamaño pequeño o medio se beneficiarán más del uso de
nuestro programa debido a la facilidad de acceso e implementación que ofrece.
\subsection{Restricciones}
Primeramente, hay que tener en cuenta un factor muy importante que es la seguridad
de los datos de los clientes según la Ley Orgánica de Protección de Datos y Garantía de
los Derechos Digitales aprobada el 23-11-2018.

Respecto a las limitaciones de hardware, no son muy elevadas ya que los ordenadores
usados por estas tiendas de ropa son lo suficientemente potentes como para ejecutar de forma fluida nuestra
aplicación, dado a que no hace uso de muchos recursos.

Las interfaces serán simples ya que este programa será usado por los clientes, empleados y
administradores de sistema, por lo que debe ser una interfaz clara e intuitiva.

Hay que tener en cuenta que esta aplicación estará instalada en cada uno de los
terminales y ordenadores de los usuarios en diferentes localizaciones geográficas, pero
con una base de datos común, de tal forma que habrá que evitar problemas de
modificación de datos, así como un posible colapso de la base de datos. Véase el apartado \ref{sec:req_pers_dat}~(\nameref{sec:req_pers_dat}).

Hay que tener en cuenta que el sistema necesitará estar apagado unas ciertas horas a
la semana (se intentará que sea cuando afecte al menor número de usuarios) para la
realización de control y auditoría.

En cuanto al lenguaje de programación las restricciones que se impondrán serán las restricciones propias de java y de la \gls{jvm}.
Los protocolos de comunicación serán sencillos ya que la única comunicación va a ser
de cada terminal con la base de datos que se hará a través de internet.



\subsection{Supuestos y dependencias}
Respecto a los cambios esperados, considerados el hecho de que la tienda de ropa a
la que vendemos el producto no sea compatible con la comunicación a través de
internet. En ese caso, dejarían de ser nuestro target.

Al utilizar java como lenguaje de programación nos podemos asegurar de que la solución funcione en cualquier sistema operativo ya que java es un lenguaje multiplataforma, lo que nos permite más flexibilidad a la hora de desarrollar el programa y hacer cambios al mismo. La \gls{bd} se puede instalar en sistemas Linux y Windows, ya que son los sistemas soportados por \gls{sql} de forma nativa sin necesidad de utilizar contenedores externos, y ya que son los sistemas mayoritariamente usados para servidores.

\subsection{Requisitos futuros}
Esta primera versión el administrador del sistema tendrá un papel fundamental en
muchas situaciones, pero cuando el proceso tenga más madurez y pase por una serie
de revisiones se obtendrán módulos más claros con roles bien definidos.

Algunos casos de uso cambiarán, y se optimizarán los módulos creados en función de
las necesidades del cliente, pero la base funcional no cambiará ya que se ha
delimitado el software a los actores y funcionalidades definidas en este documento.
